\documentclass[a4paper,11pt]{article}
\usepackage[T1]{fontenc}
\usepackage[utf8]{inputenc}
\usepackage[ngerman]{babel}
\usepackage{amsmath}
\usepackage{geometry}
\geometry{margin=1.5cm}

\title{Computer Engineering -- Submission due 27.10.2020}
\author{Jan Vogt, Yannik Köllmann - in gleichen Teilen}
\date{\today}

\begin{document}

    \maketitle

    \section{Basics of Logic Design}

    \subsection{Boolean Algebra}

    \textbf{Tasks}

    \begin{enumerate}
        \item Prove $A + \overline{A} = 1$ using perfect induction. Apply only one axiom per step and name it.

        \textbf{Beweis durch perfekte Induktion:}

        \noindent
        \textbf{Fall 1: $A = 0$}
        \begin{align*}
            A + \overline{A} &= 0 + \overline{0} \\
            &= 0 + 1 \quad \text{(Axiom: NOT)} \\
            &= 1 \quad \text{(Axiom: OR/AND)}
        \end{align*}

        \noindent
        \textbf{Fall 2: $A = 1$}
        \begin{align*}
            A + \overline{A} &= 1 + \overline{1} \\
            &= 1 + 0 \quad \text{(Axiom: NOT)} \\
            &= 1 \quad \text{(Axiom: OR/AND)}
        \end{align*}

        \noindent
        Da für beide möglichen Werte von $A$ das Ergebnis 1 ist, gilt $A + \overline{A} = 1$.

        \item Prove $A \cdot A = A$ using perfect induction. In every step apply only a single axiom. State which axiom you are using.

        \textbf{Beweis durch perfekte Induktion:}

        \noindent
        \textbf{Fall 1: $A = 0$}
        \begin{align*}
            A \cdot A &= 0 \cdot 0 \\
            &= 0 \quad \text{(Axiom: OR/AND)}
        \end{align*}

        \noindent
        \textbf{Fall 2: $A = 1$}
        \begin{align*}
            A \cdot A &= 1 \cdot 1 \\
            &= 1 \quad \text{(Axiom: OR/AND)}
        \end{align*}

        \noindent
        Da für beide möglichen Werte von $A$ gilt $A \cdot A = A$, ist die Aussage bewiesen.

        \item Prove $\overline{A + B} = \overline{A} \cdot \overline{B}$ using perfect induction. In every step apply only a single axiom. State which axiom you are using.

        \textbf{Beweis durch perfekte Induktion:}

        \noindent
        \textbf{Fall 1: $A = 0, B = 0$}
        \begin{align*}
            \overline{A + B} &= \overline{0 + 0} \\
            &= \overline{0} \quad \text{(Axiom: OR/AND)} \\
            &= 1 \quad \text{(Axiom: NOT)}
        \end{align*}
        \begin{align*}
            \overline{A} \cdot \overline{B} &= \overline{0} \cdot \overline{0} \\
            &= 1 \cdot 1 \quad \text{(Axiom: NOT)} \\
            &= 1 \quad \text{(Axiom: OR/AND)}
        \end{align*}

        \noindent
        \textbf{Fall 2: $A = 0, B = 1$}
        \begin{align*}
            \overline{A + B} &= \overline{0 + 1} \\
            &= \overline{1} \quad \text{(Axiom: OR/AND)} \\
            &= 0 \quad \text{(Axiom: NOT)}
        \end{align*}
        \begin{align*}
            \overline{A} \cdot \overline{B} &= \overline{0} \cdot \overline{1} \\
            &= 1 \cdot 0 \quad \text{(Axiom: NOT)} \\
            &= 0 \quad \text{(Axiom: OR/AND)}
        \end{align*}

        \noindent
        \textbf{Fall 3: $A = 1, B = 0$}
        \begin{align*}
            \overline{A + B} &= \overline{1 + 0} \\
            &= \overline{1} \quad \text{(Axiom: OR/AND)} \\
            &= 0 \quad \text{(Axiom: NOT)}
        \end{align*}
        \begin{align*}
            \overline{A} \cdot \overline{B} &= \overline{1} \cdot \overline{0} \\
            &= 0 \cdot 1 \quad \text{(Axiom: NOT)} \\
            &= 0 \quad \text{(Axiom: OR/AND)}
        \end{align*}

        \noindent
        \textbf{Fall 4: $A = 1, B = 1$}
        \begin{align*}
            \overline{A + B} &= \overline{1 + 1} \\
            &= \overline{1} \quad \text{(Axiom: OR/AND)} \\
            &= 0 \quad \text{(Axiom: NOT)}
        \end{align*}
        \begin{align*}
            \overline{A} \cdot \overline{B} &= \overline{1} \cdot \overline{1} \\
            &= 0 \cdot 0 \quad \text{(Axiom: NOT)} \\
            &= 0 \quad \text{(Axiom: OR/AND)}
        \end{align*}

        \noindent
        Da für alle vier möglichen Wertekombinationen von $A$ und $B$ gilt $\overline{A + B} = \overline{A} \cdot \overline{B}$, ist die Aussage bewiesen.

        \item Simplify $\overline{A}(A + B) + (B + A)(A + \overline{B})$. State the law used in each step.

        \textbf{Vereinfachung:}
        \begin{align*}
            \overline{A}(A + B) + (B + A)(A + \overline{B}) &= \overline{A}(A + B) + (A + B)(A + \overline{B}) \quad \text{(Commutativity)} \\
            &= \overline{A}(A + B) + [A + (B \cdot \overline{B})] \quad \text{(Distributivity)} \\
            &= \overline{A}(A + B) + [A + 0] \quad \text{(Inverse)} \\
            &= \overline{A}(A + B) + A \quad \text{(Identity)} \\
            &= \overline{A} \cdot A + \overline{A} \cdot B + A \quad \text{(Distributivity)} \\
            &= 0 + \overline{A} \cdot B + A \quad \text{(Inverse)} \\
            &= \overline{A} \cdot B + A \quad \text{(Identity)} \\
            &= A + \overline{A} \cdot B \quad \text{(Commutativity)} \\
            &= (A + \overline{A}) \cdot (A + B) \quad \text{(Distributivity)} \\
            &= 1 \cdot (A + B) \quad \text{(Inverse)} \\
            &= A + B \quad \text{(Identity)}
        \end{align*}
    \end{enumerate}

    \subsection{Wires and Gates}

    \textbf{Tasks}

    \begin{enumerate}
        \item Complete Table 2.2.1.

        \textbf{Vervollständigte Wahrheitstabelle:}

        \begin{center}
            \begin{tabular}{|c|c|c|c|c|c|}
                \hline
                $A$ & $B$ & $C$ & $D$ & $E$ & $F$ \\
                \hline
                0 & 0 & 0 & 0 & 0 & 0 \\
                \hline
                0 & 0 & 1 & 0 & 1 & 0 \\
                \hline
                0 & 1 & 0 & 0 & 1 & 0 \\
                \hline
                0 & 1 & 1 & 0 & 1 & 1 \\
                \hline
                1 & 0 & 0 & 0 & 1 & 0 \\
                \hline
                1 & 0 & 1 & 0 & 1 & 1 \\
                \hline
                1 & 1 & 0 & 0 & 1 & 1 \\
                \hline
                1 & 1 & 1 & 1 & 1 & 0 \\
                \hline
            \end{tabular}
        \end{center}

        \item Formulate function $\mathcal{F}(A, B, C) = (D, E, F)$ through Boolean equations, i.e., find Boolean equations which encode the provided textual descriptions.

        \textbf{Boolean-Gleichungen:}
        \begin{align*}
            D &= A \cdot B \cdot C \\
            E &= A + B + C \\
            F &= \overline{A} \cdot B \cdot C + A \cdot \overline{B} \cdot C + A \cdot B \cdot \overline{C}
        \end{align*}

        \item Design a combinational circuit in CircuitVerse that implements the function $\mathcal{F}$. Label the inputs (A, B and C) and outputs (D, E, F). \begin{verbatim}(siehe ./src)\end{verbatim}

        \item If not already done in the previous task: Design a similar circuit which only uses two-input gates. \begin{verbatim}(siehe ./src)\end{verbatim}
    \end{enumerate}

    \subsection{Universal Gates}

    \textbf{Tasks}

    \begin{enumerate}
        \item Prove the universality of \{NOR\} using Boolean equations.

        \textbf{Beweis der Universalität von NOR:}

        Um zu zeigen, dass NOR universal ist, müssen wir beweisen, dass alle grundlegenden logischen Operationen (NOT, AND, OR) nur mit NOR-Gattern ausgedrückt werden können.

        \noindent
        \textbf{NOT:}
        $$\overline{A + A} = \overline{A} \quad \text{(Idempotency)}$$

        \noindent
        \textbf{OR:}
        \begin{align*}
            \overline{\overline{A + B}} &= \overline{\overline{A} \cdot \overline{B}} \quad \text{(DeMorgan)} \\
            &= \overline{\overline{A}} + \overline{\overline{B}} \quad \text{(DeMorgan)} \\
            &= A + B \quad \text{(Double Complement)}
        \end{align*}

        \noindent
        \textbf{AND:}
        \begin{align*}
            \overline{\overline{A} + \overline{B}} &= \overline{\overline{A}} \cdot \overline{\overline{B}} \quad \text{(DeMorgan)} \\
            &= A \cdot B \quad \text{(Double Complement)}
        \end{align*}

        \noindent
        Da alle grundlegenden Operationen (NOT, OR, AND) nur mit NOR-Gattern ausgedrückt werden können, ist \{NOR\} universal.

        \item In CircuitVerse, implement the logical operations AND, OR and NOT using only two-input NOR gates. \begin{verbatim}(siehe ./src)\end{verbatim}
    \end{enumerate}

    \subsection{Equality Comparator}

    \textbf{Tasks}

    \begin{enumerate}
        \item Implement an equality comparator for the two 4-bit inputs $A_{[3:0]}$ and $B_{[3:0]}$ in CircuitVerse.

        \begin{verbatim}(siehe ./src)\end{verbatim}

        \item Showcase your design by running a simulation with the following inputs:
        \begin{enumerate}
            \item $A_{[3:0]} = 1011_2$ and $B_{[3:0]} = 1001_2$, and
            \item $A_{[3:0]} = 1101_2$ and $B_{[3:0]} = 1101_2$.
        \end{enumerate}

        \begin{verbatim}(siehe ./src)\end{verbatim}
    \end{enumerate}

\end{document}