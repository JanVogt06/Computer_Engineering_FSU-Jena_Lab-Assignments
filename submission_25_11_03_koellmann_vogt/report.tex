\documentclass[a4paper,11pt]{article}
\usepackage[T1]{fontenc}
\usepackage[utf8]{inputenc}
\usepackage[ngerman]{babel}
\usepackage{amsmath}
\usepackage{geometry}
\usepackage{circuitikz}
\usepackage{tikz}
\geometry{margin=1.5cm}

\title{Computer Engineering -- Submission due 27.10.2020}
\author{Jan Vogt, Yannik Köllmann - in gleichen Teilen}
\date{\today}

\begin{document}

    \maketitle

    \section{Combinational Building Blocks}

    \subsection{Ripple-Carry Adder}

    \textbf{Tasks}

    \begin{enumerate}
        \item Draw the schematic of the four-bit ripple-carry adder as a sequence of full adders.

        \textbf{Lösung:}

        Ein 4-Bit Ripple-Carry Adder besteht aus vier Full Adderns, die in Reihe geschaltet sind. Der Carry-Ausgang jedes Full Adders wird mit dem Carry-Eingang des nächsthöheren Full Adders verbunden.

        \begin{center}
        \begin{circuitikz}[scale=1.0, transform shape]
            % Full Adder 0 (LSB)
            \draw (0,0) node[draw, minimum width=2cm, minimum height=2.5cm, align=center] (FA0) {Full\\Adder\\0};
            \draw (FA0.north) ++(0,0.3) node[above] {$S_0$};
            \draw (FA0.west) ++(-0.3,0.5) node[left] {$A_0$};
            \draw (FA0.west) ++(-0.3,-0.5) node[left] {$B_0$};
            \draw (FA0.south) ++(0,-0.3) node[below] {$C_{in}$};
            \draw (FA0.east) ++(0.3,0) node[right] {$C_0$};
            
            % Full Adder 1
            \draw (4,0) node[draw, minimum width=2cm, minimum height=2.5cm, align=center] (FA1) {Full\\Adder\\1};
            \draw (FA1.north) ++(0,0.3) node[above] {$S_1$};
            \draw (FA1.west) ++(-0.3,0.5) node[left] {$A_1$};
            \draw (FA1.west) ++(-0.3,-0.5) node[left] {$B_1$};
            \draw (FA1.east) ++(0.3,0) node[right] {$C_1$};
            
            % Full Adder 2
            \draw (8,0) node[draw, minimum width=2cm, minimum height=2.5cm, align=center] (FA2) {Full\\Adder\\2};
            \draw (FA2.north) ++(0,0.3) node[above] {$S_2$};
            \draw (FA2.west) ++(-0.3,0.5) node[left] {$A_2$};
            \draw (FA2.west) ++(-0.3,-0.5) node[left] {$B_2$};
            \draw (FA2.east) ++(0.3,0) node[right] {$C_2$};
            
            % Full Adder 3 (MSB)
            \draw (12,0) node[draw, minimum width=2cm, minimum height=2.5cm, align=center] (FA3) {Full\\Adder\\3};
            \draw (FA3.north) ++(0,0.3) node[above] {$S_3$};
            \draw (FA3.west) ++(-0.3,0.5) node[left] {$A_3$};
            \draw (FA3.west) ++(-0.3,-0.5) node[left] {$B_3$};
            \draw (FA3.east) ++(0.3,0) node[right] {$C_{out}$};
            
            % Connections - Inputs A
            \draw[->] (FA0.west) ++(-0.5,0.5) -- (FA0.west |- FA0.north west) ++ (0,-0.4);
            \draw[->] (FA1.west) ++(-0.5,0.5) -- (FA1.west |- FA1.north west) ++ (0,-0.4);
            \draw[->] (FA2.west) ++(-0.5,0.5) -- (FA2.west |- FA2.north west) ++ (0,-0.4);
            \draw[->] (FA3.west) ++(-0.5,0.5) -- (FA3.west |- FA3.north west) ++ (0,-0.4);
            
            % Connections - Inputs B
            \draw[->] (FA0.west) ++(-0.5,-0.5) -- (FA0.west |- FA0.south west) ++ (0,0.4);
            \draw[->] (FA1.west) ++(-0.5,-0.5) -- (FA1.west |- FA1.south west) ++ (0,0.4);
            \draw[->] (FA2.west) ++(-0.5,-0.5) -- (FA2.west |- FA2.south west) ++ (0,0.4);
            \draw[->] (FA3.west) ++(-0.5,-0.5) -- (FA3.west |- FA3.south west) ++ (0,0.4);
            
            % Connections - Sum outputs
            \draw[->] (FA0.north) -- ++(0,0.5);
            \draw[->] (FA1.north) -- ++(0,0.5);
            \draw[->] (FA2.north) -- ++(0,0.5);
            \draw[->] (FA3.north) -- ++(0,0.5);
            
            % Connections - Carry chain
            \draw[->] (FA0.south) -- ++(0,-0.5);
            \draw[-] (FA0.east) -- ++(0.5,0) |- (FA1.south);
            \draw[-] (FA1.east) -- ++(0.5,0) |- (FA2.south);
            \draw[-] (FA2.east) -- ++(0.5,0) |- (FA3.south);
            \draw[->] (FA3.east) -- ++(0.5,0);
            
        \end{circuitikz}
        \end{center}

        \textbf{Funktionsweise:}
        \begin{itemize}
            \item Jeder Full Adder addiert zwei Bits ($A_i$ und $B_i$) und einen Carry-Eingang ($C_{i-1}$).
            \item Der Carry-Ausgang eines Full Adders wird als Carry-Eingang für den nächsten Full Adder verwendet.
            \item Der erste Full Adder (FA0) erhält den initialen Carry-Eingang $C_{in}$ (meist 0).
            \item Der letzte Full Adder (FA3) erzeugt den finalen Carry-Ausgang $C_{out}$.
            \item Die Summen-Bits $S_3 S_2 S_1 S_0$ bilden das 4-Bit Ergebnis.
        \end{itemize}

        \textbf{Nachteil:} Die Verzögerung beträgt $t_{RCA} = N \cdot t_{FA}$, da der Carry durch alle $N$ Full Adder "ripple" (durchlaufen) muss.
    \end{enumerate}

\end{document}