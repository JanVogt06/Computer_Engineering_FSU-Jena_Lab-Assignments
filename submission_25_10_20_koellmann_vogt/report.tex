\documentclass[a4paper,11pt]{article}
\usepackage[T1]{fontenc}
\usepackage[utf8]{inputenc}
\usepackage[ngerman]{babel}
\usepackage{geometry}
\geometry{margin=1.5cm}

\title{Computer Engineering -- Submission due 25.10.2020}
\author{}
\date{\today}

\begin{document}

\maketitle

\section{Data Encodings}

\subsection{Remote Access}

\textbf{Tasks}

\begin{enumerate}
\item Run the three commands \texttt{whoami}, \texttt{hostname}, and \texttt{lscpu} on one of the systems.

\textbf{whoami - Benutzerinformationen}

\textbf{Person 1:}
\begin{verbatim}
ce42@elaine00:~$ whoami
ce42
\end{verbatim}

\textbf{Person 2:}
\begin{verbatim}
ce43@elaine00:~$ whoami
ce43
\end{verbatim}

\textbf{hostname - Rechnername}

\begin{verbatim}
ce42@elaine00:~$ hostname
elaine00
\end{verbatim}

\textbf{lscpu - CPU-Informationen}

\begin{verbatim}
ce42@elaine00:~$ (lscpu | head -n 5; echo "---"; lscpu | tail -n 5)
Architecture:                         x86_64
CPU op-mode(s):                       32-bit, 64-bit
Address sizes:                        48 bits physical, 48 bits virtual
Byte Order:                           Little Endian
CPU(s):                               24
---
Vulnerability Spec store bypass:      Mitigation; Speculative Store Bypass disabled via prctl
Vulnerability Spectre v1:             Mitigation; usercopy/swapgs barriers and __user pointer sanitization
Vulnerability Spectre v2:             Mitigation; Enhanced / Automatic IBRS; IBPB conditional; STIBP always-on; RSB filling; PBRSB-eIBRS Not affected; BHI Not affected
Vulnerability Srbds:                  Not affected
Vulnerability Tsx async abort:        Not affected
\end{verbatim}

\item Print the contents of the file \texttt{welcome.txt} in your home directory.

\textbf{Inhalt von welcome.txt:}
\begin{verbatim}
Welcome to the Computer Engineering class! My waveform told a long story and 
ended with 'It works.'
\end{verbatim}

\end{enumerate}

\subsection{C/C++ Data Types}

\textbf{Tasks}

\begin{enumerate}
\item Write C/C++ code to print the binary representations of the following variables. \begin{verbatim}(view ./src)\end{verbatim}
\item For each of the variables, briefly explain why you obtained the resulting bits! (Siehe Erklärungen oben)

\textbf{Binäre Repräsentationen:}

\noindent
\textbf{1: 1:} \texttt{00000001} -- Dezimalwert 1.

\noindent
\textbf{2: 255:} \texttt{11111111} -- Maximalwert für 8-bit unsigned char ($2^8 - 1$).

\noindent
\textbf{3: 255 + 1:} \texttt{00000000} -- Integer Overflow: 9. Bit wird abgeschnitten.

\noindent
\textbf{4: 0xA1:} \texttt{10100001} -- Hexadezimal: A = 1010, 1 = 0001.

\noindent
\textbf{5: 0b1001011:} \texttt{01001011} -- Binärliteral auf 8 Bits aufgefüllt.

\noindent
\textbf{6: 'H':} \texttt{01001000} -- ASCII-Wert 72.

\noindent
\textbf{7: -4:} \texttt{11111100} -- Zweierkomplement: +4 invertieren und +1.

\noindent
\textbf{8: 1u << 11:} \texttt{00000000000000000000100000000000} -- Linksshift der 1 (unsigned integer) um 11 Positionen ($2^{11}$).

\noindent
\textbf{9: l\_data8 << 21:} \texttt{00000000000000000000000000000000} -- Overflow: Bit auf Position 32.

\noindent
\textbf{10: 0xFFFFFFFF >> 5:} \texttt{00000111111111111111111111111111} -- Hexadezimalzahl - alle Bits sind 1. Rechtsshift um 5, linke Bits werden 0.

\noindent
\textbf{11: 0b1001 \^{} 0b01111:} \texttt{00000000000000000000000000000110} -- XOR: 01001 \^{} 01111 = 00110.

\noindent
\textbf{12: \~{}0b1001:} \texttt{11111111111111111111111111110110} -- NOT invertiert alle 32 Bits.

\noindent
\textbf{13: 0xF0 \& 0b1010101:} \texttt{00000000000000000000000001010000} -- AND: 11110000 \& 01010101 = 01010000.

\noindent
\textbf{14: 0b001 | 0b101:} \texttt{00000000000000000000000000000101} -- OR: 001 | 101 = 101.

\noindent
\textbf{15: 7743:} \texttt{00000000000000000001111000111111} -- Dezimal zu Binär.

\noindent
\textbf{16: -7743:} \texttt{11111111111111111110000111000001} -- Zweierkomplement: invertieren und +1.
\end{enumerate}

\end{document}