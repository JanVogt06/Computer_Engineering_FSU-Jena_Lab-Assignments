\documentclass[a4paper,11pt]{article}
\usepackage[T1]{fontenc}
\usepackage[utf8]{inputenc}
\usepackage[ngerman]{babel}
\usepackage{amsmath}
\usepackage{geometry}
\usepackage{graphicx}
\usepackage{float}
\geometry{margin=1.5cm}

\title{Computer Engineering -- Submission due 01.12.2025}
\author{Jan Vogt, Yannik Köllmann - in gleichen Teilen}
\date{\today}

\begin{document}

\maketitle
Alle Dateien in src zu finden.

\section{Berechnung der k-Werte für Clock-Signale}

Das \texttt{clock}-Modul togglet bei jeder positiven Flanke des Rollover-Signals. Ein vollständiger Taktzyklus benötigt daher 2 Rollovers. Mit $f_{\text{input}} = 50\,\text{MHz}$ gilt:

\begin{equation}
f_{\text{output}} = \frac{f_{\text{input}}}{2 \cdot k} \quad \Rightarrow \quad k = \frac{f_{\text{input}}}{2 \cdot f_{\text{output}}}
\end{equation}

\subsection{10 Hz Clock}
\begin{equation}
k_{10} = \frac{50.000.000}{2 \cdot 10} = \frac{50.000.000}{20} = 2.500.000
\end{equation}
Minimales $N$: $\lceil \log_2(2.500.000) \rceil = 22$ Bits

\subsection{1 Hz Clock}
\begin{equation}
k_{1} = \frac{50.000.000}{2 \cdot 1} = \frac{50.000.000}{2} = 25.000.000
\end{equation}
Minimales $N$: $\lceil \log_2(25.000.000) \rceil = 25$ Bits

\subsection{0.1 Hz Clock}
\begin{equation}
k_{0.1} = \frac{50.000.000}{2 \cdot 0{,}1} = \frac{50.000.000}{0{,}2} = 250.000.000
\end{equation}
Minimales $N$: $\lceil \log_2(250.000.000) \rceil = 28$ Bits

\end{document}